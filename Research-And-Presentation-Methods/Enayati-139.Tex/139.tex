\documentclass{book}
\begin{document}
\begin{flushright}
\texttt{QUANTITATIVE DATA GATHERING AND ANALYSIS ON THE NET}
\hspace*{0.5cm}
\textbf{139}
\end{flushright}
\vspace*{0.6cm}
\\\hspace*{0.5cm}\\• Total number of people online by continent
\\\hspace*{0.5cm}(http://www.nua,ie/surveys/how_many_online/index.html)

\\\hspace*{0.5cm}\\• A weather-type map showing the latency or time delays experienced on
\\\hspace*{0.5cm}the Internet from an historic or real-time view at
\\\hspace*{0.5cm}(http;//www.mids.org/weather/)

\\\hspace*{0.5cm}\\• Statistics on the so-called digital divide, or the current state of
\\\hspace*{0.5cm}differentiated use of the Net by groups from different socioeconomic
\\\hspace*{0.5cm}backgrounds, gender, and race at (http://www.internetpublicpolicy.com/digdiv.cfm)

\\\hspace*{0.5cm}\\• An updated list of recent surveys completed on Net behavior and activity
\\\hspace*{0.5cm}by the IDC Corporation. This list links to worldwide information sources
\\\hspace*{0.5cm}covering a wide range of issues, although there seems to be an emphasis on
\\\hspace*{0.5cm}e-commerce applications. (http://www.nua.ie/surveys/index.cgi)\\

From these statistics we can gather an overview of Net usage, but usually the e-researcher is more interested in activity on particular sites. For this type of data collection, we move to a discussion of Web site analytics.\\


\begin{flushleft}
\texttt{WEB SITE ANALYTICS OR e-METRICS}
\hspace*{0.5cm}
\end{flushleft}
\vspace*{0.5cm}
The explosion of programming and interest in the provision of online services and access to online resources creates a new arena for e-research. Many questions have emerged as a result. Who is using the site? What resources are they utilizing? How long are they spending on each component of a site? What are participants' perceptions of the value of the site? What suggestions for site improvement do users have? Are usage patterns different between new and experienced site users?
\\\hspace*{0.5cm}There is obviously no single research tool or methodology that provides answers to these and many other important research questions. The traditional means of assessing participants' perceptions, suggestions, and concerns through survey or interview research, coupled with evaluations of outcomes has provided answers to some of these questions. However, for other questions, the online environment itself provides a wealth of relevant data. The analysis of this data is a subset of the emerging (and somewhat over-hyped) field of study known as \emph{data mining}. Two Crows Consulting describes data mining as "a combination of machine learning, statistical analysis, modeling techniques and database technology. Data mining finds patterns and subtle relationships in data and infers rules that allow the prediction of future results" (Two Crows Corporation, 1999). Clearly, the research possibilities for such analysis is great, A few of these benefits include the capacity to identify which activities and resources were used most (and least) frequently, the ability to record the length of time participants spend individually and on average using a particular resource, the ability to adapt the activities in response to data gathered on user behaviors, and the capacity to identify individual and group problems when accessing particular pages. The data from Net-based environments are, in one important sense, more accessible to the e-researcher than data from equivalent non-networked environments in that all interactivity, postings, and navigation are automatically recorded by the programs that create
\end{document} 